%
\begin{tikzpicture}[remember picture,overlay]
    \node [yshift=-1.3cm,color=black!40] at (current page.north)
        {%
        \hfill%
        Seminar \enquote{Das Wortproblem} im
        WS~2014/15 an der Universität Regensburg%
        \hfill\mbox{}%
        };
\end{tikzpicture}

\vspace*{-0.5cm}
\begin{center}
    \Large Der Einbettungssatz von Higman
\end{center}

\medskip\noindent
J. Prem (\href{mailto:Johannes.Prem@stud.uni-regensburg.de}%
{\texttt{Johannes.Prem@stud.uni-regensburg.de}})
\hfill
16.~Dezember~2014
\\[-8pt]
\rule{\textwidth}{0.4pt}

\begin{center}
    \parbox{0.82\textwidth}{%
        Higmans Einbettungssatz schafft eine Verbindung zwischen der Welt der
        Gruppen und grundlegender Berechenbarkeitstheorie (z.\,B. im Sinne von
        Turingmaschinen). Im Folgenden geben wir eine Beweisskizze dieses
        Satzes (Rotman\cite{bookc:rotman95} folgend), die auf dem Beweis
        von Aanderaa\cite{paper:aanderaa73} beruht.
    }
\end{center}
