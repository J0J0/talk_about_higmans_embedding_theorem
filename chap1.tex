\chapter{Der Einbettungssatz von Higman}
\begin{thDef}[rekursiv präsentiert]
    Sei $R$ eine Gruppe. Dann ist $R$ \emph{rekursiv präsentiert},
    falls $R$ endlich erzeugt ist und es eine Präsentation
    $R = \gpres{u_1,\dots,u_m}{E}$ von~$R$ gibt, so dass
    jedes $\omega\in E$ ein positives Wort auf~$\{u_1,\dots,u_m\}$
    und die Menge~$E$ rekursiv aufzählbar ist.
\end{thDef}

\begin{thSatz}[Einbettungssatz von Higman]
    \label{ch1:higman}
    %
    Jede rekursiv präsentierte Gruppe kann in eine endlich präsentierte Gruppe
    eingebettet werden.
\end{thSatz}

Den Rest des Skripts werden wir nun damit verbringen, diesen Satz zu beweisen.

\begin{thSetup}\hfill
    \begin{itemize}
    \item 
        Sei ab jetzt eine rekursiv präsentierte Gruppe
        \[ R = \gpres{u_1,\dots,u_m}{E} \]
        fixiert und sei $U \defeq \{u_1,\dots,u_m\}$.
    \item
        Aus notationellen Gründen führen wir eine weitere Menge
        $A \defeq \{a_1,\dots,a_m\}$ von neuen Symbolen ein und bezeichnen mit
        $\tra$ den eindeutigen Monoid-Isomorphismus (vom Monoid aller
        positiven Wörter auf $\{u_1,\dots,u_m\}$ in das Monoid aller positiven
        Wörter auf $\{a_1,\dots,a_m\}$) mit $u_j\mapsto a_j$ für alle
        $j\in\setOneto{m}$. Sei $\Ea \defeq \tra(E)$.
    \item
        Sei weiter $T$~eine Turingmaschine, die $\Ea$ aufzählt,\footnote{%
            So eine Turingmaschine~$T$ existiert offenbar, denn nach
            Voraussetzung gibt es eine Turingmaschine, die $E$ aufzählt,
            und $E\rightsquigarrow\Ea$ ist nur eine Umbenennung der
            Symbole.%
        }
        sei $S = \{s_0,\dots,s_M\} \supset A$ ihr Eingabealphabet
        (wobei $s_0$ das \enquote{leeres Band}-Zeichen sei) und sei
        $Q = \{q_0,\dots,q_N\}$ ihre Zustandsmenge. Wir nehmen ohne
        Einschränkung an, dass $q_0$ der einzige Endzustand sowie $q_1$ der
        Anfangszustand von~$T$ ist.
    \end{itemize}
\end{thSetup}

\begin{thProposition}[Halbgruppe assoziiert zu~$T$ {[Post]}]
    \label{ch1:Gamma}
    %
    Es gibt eine Halbgruppe~$\Gamma$ mit der Präsentation
    \begin{itemize}
        \item Erzeuger:
                $\{q,h\} \cup S\cup Q$ (mit neuen Symbolen $q,h\notin S\cup Q$),
        \item Relationen:
                $\{ F_iq_{i,1}G_i = H_iq_{i,2}K_i \Mid i\in I \}$,
                wobei $I$ eine endliche Indexmenge ist und für alle
                $i\in I$ gilt:
                $q_{i,1},q_{i,2}\in \{q\}\cup Q$ und $F_i,G_i,H_i,K_i$ sind
                (möglicherweise leere) positive Wörter auf $S\cup\{h\}$,
    \end{itemize}
    und folgender Eigenschaft: Ist $w$ ein positives Wort auf~$S$, so
    gilt 
    \[ w\in\Ea \iff hq_1wh = q \in\Gamma  . \]
\end{thProposition}
%
Siehe Rotman\cite{bookc:rotman95}. % TODO

\pagebreak[2]
Im Folgenden seien $\Gamma$ und alle zur Präsentation gehörenden Daten
wie in~\cref{ch1:Gamma} fest gewählt. Zur Vereinfachung der Notation setzen
wir außerdem $S_+ \defeq S\cup\{h\}$.

\begin{thDef}[Variante von Boones Gruppe]
    \label{ch1:BT}
    %
    Sei die Gruppe $\BT$ gegeben durch:
    \begin{itemize}
        \item Erzeuger:
                $\{q,h\} \cup S\cup Q
                \cup \{r_i\Mid i\in I\}\cup\{x,\kappa,\tau\}$
                (mit neuen Symbolen $r_i$ und $x,\kappa,\tau$).
        \item Relationen:
                $\Delta_1 \cup \Delta_2 
                    \cup \Delta_3 \cup \Delta$,
                wobei
                \begin{align*}
                    \Delta_1 &\defeq \bigl\{ s\inv xs = x^2
                        \Mid s\in S_+ \bigr\}
                    \\[4pt]
                    \Delta_2 &\defeq \bigl\{ r_i\inv sxr_i = sx\inv
                        \Mid s\in S_+,\, i\in I \bigr\}
                    \\
                    &\qquad\cup \bigl\{
                        r_i\inv F_i^\# q_{i,1}G_ir_i = H_i^\# q_{i,2}K_i
                        \Mid i\in I \bigr\}
                    \\[4pt]
                    \Delta_3 &\defeq \bigl\{
                        \cu[\tau, q_1\inv hr_ih\inv q_1] \Mid i\in I
                        \bigr\} \cup \bigl\{
                        \cu[\tau, q_1\inv hxh\inv q_1] \bigr\}
                    \\[4pt]
                    \Delta &\defeq \bigl\{
                        \cu[\kappa, hr_ih\inv] \Mid i\in I
                        \bigr\} \cup \bigl\{
                        \cu[\kappa, hxh\inv] \bigr\}
                    \\
                    &\qquad\cup \bigl\{
                        \cu[\kappa,
                            hq\inv h\inv q_1\tau q_1\inv hqh\inv]
                    \bigr\}
                . \end{align*}
    \end{itemize}
    Dabei ist für ein Wort~$F = y_1^{\epsilon_1}\cdots y_n^{\epsilon_n}$
    auf~$S_+$ (mit $\epsilon_j\in\{\pm1\}$ für alle $j\in\setOneto n$)
    das Wort $F^\#$ definiert als $y_1^{-\epsilon_1}\cdots y_n^{-\epsilon_n}$.
\end{thDef}

\begin{thProposition}[Eigenschaften von \texorpdfstring{$\BT$}{B}]\hfill
    \label{ch1:BTproperties}
    %
    \begin{enumerate}[1.]
        \item\label{ch1:BTproperties:boone}
            Die Gruppe~$\BT$ erfüllt folgende Eigenschaft:
            Ist $w$ ein positives Wort auf~$S$, so gilt
            \[ w\in\Ea \iff \cu[\kappa, w\inv\tau w] = 1 \in\BT
            . \]

        \item\label{ch1:BTproperties:hnn}
            Es entsteht $\BT$ als eine Kette von HNN-Erweiterungen
            {\thickmuskip=10mu%
            \[ \BT_0 \leq \BT_1 \leq \hBT_2 \leq \BT_2 \leq \BT_3 \leq \BT
            , \]}
            wobei
            \begin{align*}
                \BT_0 &\defeq \gpres{x}{\emptyset}
                \\
                \BT_1 &\defeq \gadj{\BT_0}{S_+}{\Delta_1}
                \\
                \hBT_2 &\defeq \gadj{\BT_1}{Q\cup\{q\}}{\emptyset}
                         = \BT_1 \fprod \gpres{Q\cup\{q\}}{\emptyset}
                \\
                \BT_2 &\defeq \gadj{\hBT_2}{\{r_i\Mid i\in I\}}{\Delta_2}
                \\
                \BT_3 &\defeq \gadj{\BT_2}{\{\tau\}}{\Delta_3}
                \\
                \BT &\mathrel{\makebox[\widthof{$\defeq$}]{\hfill$=$}}
                        \gadj{\BT_3}{\{\kappa\}}{\Delta}
            \end{align*}
            und die stabilen Buchstaben jeweils die
            neu hinzugefügten Symbole %TODO: \ref zur Def. von \gadj..!?
            sind.
    \end{enumerate}
\end{thProposition}

\begin{thBemerkung}[Notation bei Rotman]
    Im Buch von Rotman\cite{bookc:rotman95} lauten die Bezeichnungen
    der Gruppen $\BT_3$ bzw. $\BT$ leicht anders:
    $\mathscr{B}_3'(T)$ bzw. $\mathscr{B}'(T)$. Die Gruppen
    $\hBT_2$ und $\hBT_4$ (s.\,u.) haben bei Rotman keine explizite
    Bezeichnung. Ansonsten wurde die Notation so gewählt, dass die
    Nummerierung der Gruppen mit der von Rotman übereinstimmt.
\end{thBemerkung}

\begin{proof}[Beweisskizze von \cref{ch1:BTproperties}]
    Zu~\ref{ch1:BTproperties:boone}: Wir betrachten die bei Rotman 
    in Kapitel~12 überhalb von Lemma~12.7 definierte Gruppe~$\mathscr{B}(T)$
    (wobei zu beachten ist, dass das dortige $s_{M-1}$ unserem $s_M$
    und das dortige $s_M$ unserem $h$ entspricht).
    Durch Vergleich der Präsentationen der Gruppen $\mathscr{B}(T)$
    und~$\BT$ sieht man leicht mithilfe der universellen Eigenschaft
    von durch Erzeuger und Relationen gegeben Gruppen ein, dass man
    zueinander inverse Gruppenhomomorphismen $\phi\colon\mathscr{B}(T)
    \to\BT$ und $\psi\colon\BT\to\mathscr{B}(T)$ mit
    \begin{align*}
        \phi(\tau)   &= q_1\inv hth\inv q_1
        ,&
        \psi(t)      &= h\inv q_1\tau q_1\inv h
        \\
        \phi(\kappa) &= hkh\inv
        ,&
        \psi(k)      &= h\inv\kappa h
    \end{align*}
    konstruieren kann. Das \emph{Lemma von Boone}
    (Rotman\cite[Lemma~12.7]{bookc:rotman95}) liefert dann
    (in der Notation von Rotman für $\Sigma \equiv h\inv q_1wh$):
    Für ein positives Wort~$w$ auf~$S$ gilt:
    \[ hq_1wh = q \in\Gamma
        \qiffq
        \cu[k, (h\inv q_1wh)\inv t (h\inv q_1wh)] = 1
        \in\mathscr{B}(T)
    . \]
    Unter dem Isomorphismus~$\phi$ wird die rechte Gleichung
    dann gerade zu $\cu[\kappa, w\inv\tau w] = 1 \in\BT$
    und mit \cref{ch1:Gamma} folgt die erste Behauptung.
    
    Zu~\ref{ch1:BTproperties:hnn}: Die letzten beiden Erweiterungen
    sind HNN-Erweiterungen nach \cref{ch0:simpleHNN}. Dass $\hBT_2$
    eine HNN-Erweiterung mit Basis~$\BT_2$ ist, ist klar (wähle
    $M_{\tilde q} = \emptyset$ für alle $\tilde q\in Q\cup\{q\}$).
    Weiter induziert $x\mapsto x^2$ klarerweise einen Isomorphismus
    zwischen den unendlich zyklischen Gruppen $\scriptstyle\Spann{x}_{\BT_0}$
    und $\scriptstyle\Spann{x^2}_{\BT_0}$, was zeigt, dass $\BT_1$ eine
    HNN-Erweiterung mit Basis~$\BT_0$ ist. Es bleibt zu zeigen,
    dass $\BT_2$ eine HNN-Erweiterung mit Basis~$\hBT_2$ ist.
    Dazu zeigt man (für jedes $i\in I$), dass $\{F_i^\#q_{i,1}G_i\}\cup
    S_+x$ ein freies Erzeugendensystem der von dieser Menge
    in~$\hBT_2$ erzeugten Untergruppe ist und analog für
    $\{H_i^\#q_{i,2}K_i\}\cup S_+x\inv$, womit man leicht die
    gewünschten Isomorphismen konstruiert. 
    \\
\end{proof}

\begin{thDef}[Erweiterungen von \texorpdfstring{$\BT$}{B}]
    \label{ch1:BT6fromBT}
    Sei $B \defeq \{b_1,\dots,b_m\}$ eine Menge neuer Symbole
    und seien weiter $d,\sigma$ neue Symbole. Dann definieren
    wir sukzessive Erweiterungen von~$\BT$ wie folgt:
    \begin{align*}
        \hBT_4 &\defeq \gadj{\BT}{U}{E} = \BT \fprod R
        \\
        \BT_4 &\defeq \gadj{\hBT_4}{B}{\Delta_4}
        \\
        \BT_5 &\defeq \gadj{\BT_4}{\{d\}}{\Delta_5}
        \\
        \BT_6 &\defeq \gadj{\BT_5}{\{\sigma\}}{\Delta_6}
    , \end{align*}
    wobei
    \begin{align*}
        \Delta_4 &\defeq \bigl\{
            \cu[b, u] \Mid b\in B,\, u\in U
            \bigr\} \cup \bigl\{
            \cu[b, a] \Mid b\in B,\, a\in A
            \bigr\}
        \\
        &\qquad\cup \bigl\{
            b_j\inv\kappa b_j = \kappa u_j\inv \Mid j\in\J \bigr\}
        \\[2pt]
        \Delta_5 &\defeq \bigl\{ \cu[d, \kappa] \bigr\}
            \cup \bigl\{ d\inv a_jb_j d = a_j \Mid j\in\J \bigr\}
        \\[2pt]
        \Delta_6 &\defeq \bigl\{ \cu[\sigma, \kappa],\;
            \sigma\inv\tau\sigma = \tau d \bigr\}
            \cup \bigl\{ [\sigma, a] \Mid a\in A \bigr\}
    . \end{align*}
\end{thDef}

\pagebreak[2]
\begin{thProposition}[\texorpdfstring{$R\leq \BT_6$}%
                        {R als Untergruppe von B6}]\hfill
    \label{ch1:RinBT6}
    %
    \begin{enumerate}[1.]
        \item 
            Es entsteht $\BT_6$ aus $\hBT_4$ als eine Kette von
            HNN-Erweiterungen wie folgt:
            {\thickmuskip=10mu%
            \[ \hBT_4 \leq \BT_4 \leq \BT_5 \leq \BT_6  . \]}
        \item
            Es ist $R$ (vermöge der kanonischen Abbildung
            $R\to\BT_6$) eine Untergruppe von~$\BT_6$.
    \end{enumerate}
\end{thProposition}

\begin{thLemma}\label{ch1:freesubgroupsinBT}
    Es sind $A\cup\{\tau\}$ bzw. $A\cup\{\kappa\}$ freie
    Erzeugendensysteme der von diesen Mengen in~$\BT_3$ bzw.
    $\BT$ erzeugten Untergruppen.
\end{thLemma}
%
\begin{proof}
    Wir zeigen die erste Behauptung, die zweite folgt mit einem
    analogen Argument (siehe Rotman\cite[Lemma~12.21]{bookc:rotman95}).
    
    Nach \mycref{ch1:BTproperties:hnn} und \cref{ch0:hnnstablesfree}
    ist die von $S_+$ in $\BT_1$ erzeugte Untergruppe frei
    und mit $\BT_1 \leq \BT_3$ sowie $A\subset S_+$ folgt:
    $A$ ist ein freies Erzeugendensystem von $\Spann{A}_{\BT_3}$.
    Angenommen es gibt ein (frei reduziertes) Wort~$w$ auf~$A\cup\{\tau\}$
    mit $w=1\in\BT_3$, in dem $\tau$ vorkommt. Sei außerdem $w$ von
    minimaler Länge mit diesen Eigenschaften.
    Weil $\BT_3$ eine HNN-Erweiterung mit Basis~$\BT_2$ und stabilem
    Buchstaben~$\tau$ ist, gibt es dann nach dem Lemma von Britton einen
    $\tau$-pinch in~$w$, d.\,h. es gilt
    \[ w \equiv w_0 \, \tau^\epsilon w_1 \tau^{-\epsilon} \, w_2 \]
    für $\epsilon\in\{\pm1\}$, geeignete Wörter $w_0,w_2$ und
    ein Wort $w_1$, welches in $\BT_2$ gleich einem Wort aus
    \[ \Spann[\big]{
            \{q_1\inv hr_ih\inv q_1 \nMid i\in I\}
            \cup\{q_1\inv hxh\inv q_1\}
        }_{\BT_2}
    \]
    ist. Die Kommutator-Relationen in~$\Delta_3$ zeigen dann aber, dass
    $\tau$ in~$\BT_3$ mit $w_1$ kommutiert, womit wir
    \[ 1 = w = w_0 w_1 \tau^\epsilon \tau^{-\epsilon} w_2 = w_0w_1w_2
    \]
    in~$\BT_3$ erhalten, ein Widerspruch zur minimalen Wortlänge von~$w$.
    Also kann in einem (frei reduzierten) Wort~$w$ auf~$A\cup\{\tau\}$
    mit $w=1\in\BT_3$ kein~$\tau$ vorkommen, d.\,h. es muss schon
    $w\in A^*$ gelten. Aber wir haben oben gesehen, dass die von~$A$
    erzeugte Untergruppe frei ist, womit schon $w\equiv1$ gelten muss.
    Also gibt es keine nicht-trivialen Relationen
    in~$\Spann{A\cup\{\tau\}}_{\BT_3}$ und es folgt die Behauptung.
    \\
\end{proof}

\begin{thKorollar}\label{ch1:freesubgroupsinhBT4}
    Die Aussagen aus \cref{ch1:freesubgroupsinBT} gelten analog auch
    in~$\hBT_4$ (statt $\BT_3$ und~$\BT$).
\end{thKorollar}
%
\begin{proof}
    Wegen $\BT_3\leq\BT$ (\mycref{ch1:BTproperties:hnn}) und
    weil $\BT$ ein freier Faktor von~$\hBT_4$ ist, folgt dies
    direkt aus \cref{ch1:freesubgroupsinBT}.
    \\
\end{proof}

\begin{thLemma}\label{ch1:HNNeBT4}
    Es ist $\BT_4$ eine HNN-Erweiterung mit Basis~$\hBT_4$.
\end{thLemma}
%
\begin{proof}
    Sei $b\in B$ und sei $u\defeq\tr^B_U(b)$.
    Dann haben wir mit
    \[ M_b = \{ (u,u), (a,a) \Mid u\in U, a\in A\}
        \cup\{(\kappa,\kappa u\inv)\}
    \]
    zu zeigen, dass
    \begin{align*}
        \phi\colon\Spann{A\cup \{\kappa\}\cup U}_{\hBT_4}
        &\to
        \Spann{A\cup \{\kappa u\inv\}\cup U}_{\hBT_4}
        \\
        \kappa &\mapsto \kappa u\inv  \\
        \phi\vert_{A\cup U} &\defeq \id_{A\cup U}
    \end{align*}
    ein wohldefinierter Isomorphismus zwischen diesen (identischen!)
    Untergruppen von~$\hBT_4$ ist. Weil es aber in $\hBT_4 = \BT\fprod R$
    offenbar keine Relationen zwischen den Symbolen aus~$A\cup\{\kappa\}$
    und denen aus~$U$ gibt, sieht man sofort
    \[ \Spann{A\cup \{\kappa\}\cup U}_{\hBT_4}
        \cong \Spann{A\cup\{\kappa\}}_{\hBT_4} \fprod R
    \]
    ein. Mit \cref{ch1:freesubgroupsinhBT4} und den
    universellen Eigenschaften von freien Gruppen und
    freien Produkten folgt, dass $\phi$ ein wohldefinierter
    Gruppenhomomorphismus ist. Weiter erhält man aus
    $\kappa\mapsto\kappa u$ und ansonsten derselben
    Definition wie bei~$\phi$ einen zu $\phi$ inversen
    Homomorphismus, womit alles gezeigt ist.
    \\
\end{proof}

\begin{thLemma}\label{ch1:HNNeBT5}
    Es ist $\BT_5$ eine HNN-Erweiterung mit Basis~$\BT_4$.
\end{thLemma}
%
\begin{proof}
    Wir haben
    \[ M_d = \bigl\{ (\kappa,\kappa) \bigr\}
        \cup \bigl\{ (a_jb_j,a_j) \Mid j\in\J \bigr\}
    . \]
    Die Abbildung $\phi\colon\BT_4\to\hBT_4$, die auf
    $\BT$ die Identität ist und alle Erzeuger aus $B\cup U$
    auf~$1\in\hBT_4$ abbildet, schickt offenbar alle Relationen
    aus~$\Delta_4$ auf die triviale Relation in~$\hBT_4$, womit
    $\phi$ nach der universellen Eigeschaft von durch Erzeugern
    und Relationen gegeben Gruppen einen wohldefinierten
    Gruppenhomomorphismus definiert. Nun bildet~$\phi$ die
    Untergruppe
    \[ H\defeq\Spann[\big]{\bigl\{ \kappa \bigr\}
        \cup \bigl\{ a_jb_j \Mid j\in\J \bigr\}}_{\BT_4}
    \]
    auf die Untergruppe $\Spann{A\cup\{\kappa\}}_{\hBT_4}$
    ab, mit $\kappa\mapsto\kappa$ und $a_jb_j\mapsto a_j$
    für alle $j\in\J$. Mithilfe der universellen Eigenschaft
    der (nach \cref{ch1:freesubgroupsinhBT4}) freien Gruppe
    $\Spann{A\cup\{\kappa\}}_{\hBT_4}$ folgert man leicht,
    dass damit auch~$H$ frei von den obigen Elementen erzeugt wird.
    Aus \cref{ch1:HNNeBT4} folgt insbesondere $\hBT_4\leq\BT_4$,
    d.\,h. auch $H'\defeq\Spann{A\cup\{\kappa\}}_{\BT_4}$ ist frei
    erzeugt von $A\cup\{\kappa\}$. Mit den universellen Eigenschaften
    der freien Gruppen $H$ und~$H'$ erhalten wir sofort den
    gewünschten Isomorphismus $H\to H'$, der die Behauptung zeigt.
    \\
\end{proof}

\begin{thLemma}\label{ch1:presofAkappatau}
    Die von $A\cup\{\kappa,\tau\}$ in $\BT$ erzeugte Untergruppe
    besitzt folgende Präsentation:
    \[ \mathcal{A} \defeq
        \gpres[{[\big]}]{A\cup\{\kappa,\tau\}}%
                   {\{ \cu[\kappa, \omega\inv\tau\omega]
                        \nMid \omega\in\Ea \}}
    . \]
\end{thLemma}

\begin{thLemma}\label{ch1:wordsinAkappatau}
    Zu jedem $w'\in\mathcal A$ (mit $\mathcal A$ aus
    \cref{ch1:presofAkappatau}) gibt es ein
    gibt es ein Wort~$w$ auf $A\cup\{\kappa,\tau\}$
    mit $w'=w$ in~$\mathcal A$, so dass $w$ kein
    Teilwort der Form $\tau^\epsilon\omega\kappa^\eta$
    mit $\epsilon,\eta\in\{\pm1\}$ und $\omega\in\Ea$ enthält.
\end{thLemma}
%
\begin{proof}
    Sei $w'$ ein (frei reduziertes) Wort aus~$\mathcal A$
    und seien $y_1,\dots,y_n\in A\cup\{\kappa,\tau\}$,
    $\epsilon_1,\dots,\epsilon_n\in\{\pm1\}$
    mit
    \[ w' \equiv
        y_1^{\epsilon_1}y_2^{\epsilon_2}\cdots y_n^{\epsilon_n}
    . \]
    Die Zahl
    \[ c(w') \defeq \sum_{j=1}^n \delta_{y_j,\tau}
                        \sum_{\ell=j+1}^n \delta_{y_\ell,\kappa}
        \;\in\,\N
    \]
    (wobei $\delta_{y,y'} = 1$ falls $y=y'$ und $\delta_{y,y'} = 0$ sonst)
    beschreibt, wie oft~$\tau$ vor~$\kappa$ vorkommt.
    Sei nun $w\in (A\cup\{\kappa,\tau\})^*$ mit $w'=w$ in~$\mathcal A$,
    so dass $c(w)$ minimal ist. Wir zeigen nun, dass $w$ das gesuchte
    Wort ist.
    
    Angenommen, es gibt $\epsilon,\tau\in\{\pm1\}$ und $\omega\in\Ea$,
    so dass
    \[ w \equiv w_0 \, \tau^\epsilon\omega\kappa^\eta \, w_1 \]
    (für geeignete Wörter $w_0,w_1$) gilt. Nach Definition
    von $\mathcal A$ gilt $\cu[\kappa, \omega\inv\tau\omega] = 1
    \in\mathcal A$ und durch Ausschreiben dieser Gleichung und einfache
    Umformungen erhalten wir:
    \[ \tau^\epsilon\omega\kappa^\eta
        = \omega\kappa^\eta\omega\inv\tau^\epsilon\omega
        \in\mathcal A
    . \]
    Es folgt
    \[ w = w_0 \, \omega\kappa^\eta\omega\inv\tau^\epsilon\omega \, w_1
        \eqdef w''
    \]
    in~$\mathcal A$, aber offenbar gilt dann $c(w'') < c(w)$, im
    Widerspruch zur Wahl von~$w$.
    \\
\end{proof}

\begin{proof}[Beweisskizze von \cref{ch1:wordsinAkappatau}]
    \mycref{ch1:BTproperties:boone} zeigt, dass die kanonische
    Abbildung
    \[ A\cup\{\kappa,\tau\}\to\Spann{A\cup\{\kappa,\tau\}}_{\BT} \]
    die Relationen aus $\mathcal A$ erhält und somit nach der universellen
    Eigschaft von durch Erzeuger und Relationen gegeben Gruppen einen
    Homomorphismus 
    \[ \phi\colon\mathcal{A}\to\Spann{A\cup\{\kappa,\tau\}}_{\BT} \]
    induziert. Offensichtlich ist~$\phi$ surjektiv, d.\,h. es bleibt
    die Injektivität von~$\phi$ zu zeigen. Sei also
    $w\in\mathcal A$ mit $\phi(w) \mathrel{\glqq{=}\grqq}
    w = 1 \in\Spann{A\cup\{\kappa,\tau\}}_{\BT}$.
    Nach \cref{ch1:wordsinAkappatau} können wir o.\,E. annehmen, dass
    $w$ kein Teilwort der Form $\tau^\epsilon\omega\kappa^\eta$ mit
    $\epsilon,\eta\in\{\pm1\}$ und $\omega\in\Ea$ enthält.
    
    Man kann nun zeigen, dass die Annahme $w\not\equiv 1$ zu einem
    Widerspruch führt. Der sehr technische und längliche Beweis
    kann bei Rotman\cite[Lemma~12.24]{bookc:rotman95} nachgelesen werden;
    wir skizzieren hier nur grob das Vorgehen von Rotman:
    \begin{itemize}
        \item
            Wir haben $w = 1\in\Spann{A\cup\{\kappa,\tau\}}_{\BT}$
            nach Wahl von~$w$.\\
            Angenommen, es ist $w$ \emph{nicht} das leere Wort.
            
        \item
            Zuerst folgt dann aus \cref{ch1:freesubgroupsinBT}, dass
            $w$ sowohl $\tau$ als auch $\kappa$ enthalten muss.
            
        \item
            Das Lemma von Britton \pcref{ch0:brittonslemma} liefert
            dann einen $\kappa$-pinch $\kappa^\eta V\!\kappa^{-\eta}$
            in~$w$, wobei
            \[ D\in \bigl(\{hr_ih\inv \Mid i\in I\}
                \cup\{hxh\inv,\, hq\inv h\inv q_1\tau q_1\inv hqh\inv\}
                \bigr)^*
                \qundq
                D = V \in\BT_3
            . \]
            
        \item
            Man kann nun zeigen, dass $D$ so gewählt werden kann,
            dass $D$ keinen $\tau$-pinch enthält, und außerdem
            kann man zeigen, dass $V$ keinen $\tau$-pinch enthält.
            
        \item
            Aus dem letzten Schritt kann man (unter anderem mit mehrmaliger
            Anwendung des Lemmas von Britton) herleiten, dass sowohl
            $V$ als auch~$D$ mindestens ein~$\tau$ enthalten.
            
        \item
            Indem man das letzte Vorkommen von~$\tau$ in~$V$ betrachtet
            und $V=D\in\BT_3$ benutzt, kann man zeigen: $V$~endet auf
            ein Wort $\tau^\epsilon\omega$ mit $\epsilon\in\{\pm1\}$ und
            $\omega\in\Ea$.
            
        \item
            Aus den vorherigen Schritten erhalten wir:
            $w$~enthält den $\kappa$-pinch $\kappa^\eta V\!\kappa^{-\eta}$
            und $V$ endet auf~$\tau^\epsilon\omega$, d.\,h. es gilt
            \[ w \equiv w_0 \, \kappa^\eta V\!\kappa^{-\eta} \, w_2
                 \equiv w_0 \, \kappa^\eta w_1 \,
                 \tau^\epsilon\omega\kappa^{-\eta} \, w_2
            \]
            für geeignete Wörter~$w_0,w_1,w_2$.
            Aber dies ist ein Widerspruch dazu, dass $w$ kein Teilwort
            der Form $\tau^\epsilon\omega\kappa^{-\eta}$ enthält, womit
            die Annahme falsch gewesen sein muss, d.\,h. es gilt doch schon
            \[ w\equiv 1  , \]
            und damit ist $\phi$ auch injektiv.
            \\
            \qedhere
    \end{itemize}
\end{proof}

\begin{thLemma}\label{ch1:conjugatesinBT5}
    Sei $\omega\in\Ea$ und seien $\omega_b \defeq \tr^A_B(\omega)$,
    $\omega_u \defeq \tr^A_U(\omega)$. Dann gilt in~$\BT_5$:
    \begin{align*}
        \kappa\inv \omega_b \kappa &= \omega_b\omega_u  \\
        d \omega d\inv &= \omega\omega_b
    . \end{align*}
\end{thLemma}
%
\begin{proof}
    Dies folgt leicht aus den in~$\BT_5$ gültigen Relationen.
    Die Details werden dem Leser zur Übung überlassen.
    \\
\end{proof}

\begin{thLemma}\label{ch1:HNNeBT6}
    Es ist $\BT_6$ eine HNN-Erweiterung mit Basis~$\BT_5$.
\end{thLemma}
%
\begin{proof}
    Wir haben
    \[ M_\sigma = \{ (\kappa,\kappa),\, (\tau,\tau d) \}
            \cup \{ (a,a) \Mid a\in A \}
    . \]
    Wir betrachten also
    \begin{align*}
        \phi\colon\Spann{A\cup\{\kappa,\tau\}}_{\BT_5}
        &\to \Spann{A\cup\{\kappa,\tau d\}}_{\BT_5}
        \\
        \tau &\mapsto \tau d    \\
        \phi\vert_{A\cup\{\kappa\}} &\defeq \id_{A\cup\{\kappa\}}
    \end{align*}
    und müssen zunächst zeigen, dass $\phi$ einen Gruppenhomomorphismus
    induziert. Nach \cref{ch1:presofAkappatau} (und wegen $\BT\leq\BT_5$)
    genügt es dafür zu zeigen: für alle $\omega\in\Ea$ gilt
    \[  \cu[\kappa, \omega\inv\tau d\omega]
        = \phi(\cu[\kappa, \omega\inv\tau\omega])
        = 1 \in\BT_5
    . \]
    Sei also $\omega\in\Ea$ und seien $\omega_b \defeq \tr^A_B(\omega)$,
    $\omega_u \defeq \tr^A_U(\omega)$. Dann gilt:
    \begin{align*}
        \cu[\kappa, \omega\inv\tau d\omega]
        &=
        \expandafter\bigl\cu[\kappa, \omega\inv\tau (d\omega d\inv) d\bigr]
        \overset{\scriptscriptstyle(1)}=
        \cu[\kappa, \omega\inv\tau \omega\omega_b d]
        \\
        &= \bigl(\omega\inv\tau\omega\omega_b d\kappa\bigr)\inv
            \kappa \omega\inv\tau\omega\omega_b d
        \\
        &\overset{\scriptscriptstyle(2)}=
        \bigl(\omega\inv\tau\omega\kappa(\kappa\inv\omega_b\kappa) d\bigr)\inv
            \omega\inv\tau\omega\kappa\omega_b d
        \\
        &\overset{\scriptscriptstyle(3)}=
        \bigl(\omega\inv\tau\omega\kappa\omega_b\omega_u d\bigr)\inv
            \omega\inv\tau\omega\kappa\omega_b d
        \\
        &\overset{\scriptscriptstyle(4)}= 1
    , \end{align*}
    wobei wir Folgendes benutzt haben:
    \begin{enumerate}[(1), itemsep=0pt, topsep=0.5\baselineskip]
        \item
            \cref{ch1:conjugatesinBT5}.
        \item
            Es kommutiert $\kappa$ nach Definition von~$\BT_5$ mit~$d$
            und nach \mycref{ch1:BTproperties:boone} mit
            $\omega\inv\tau\omega$.
        \item
            \cref{ch1:conjugatesinBT5}.
        \item
            Nach Konstruktion gilt gerade $\omega_u\in E$, wobei
            $E$ die Menge der Relationen von~$R$ war, und diese Relationen
            sind in~$\BT_5$ ebenfalls gültig, d.\,h. $\omega_u = 1\in\BT_5$.
    \end{enumerate}
    Mit der universellen Eigenschaft von durch Erzeugern und Relationen
    gegebenen Gruppen sieht man sofort ein, dass
    \begin{align*}
        \psi\colon \BT_5 &\to \BT_5
        \\
        \mathllap{B\cup U\cup\{d\}} \ni y &\mapsto 1     \\
        \psi\vert_{\BT} &\defeq \id_{\BT}
    \end{align*}
    einen wohldefinierten Gruppenhomomorphismus induziert. Außerdem gilt
    offenbar $\im(\psi) \subset \Spann{A\cup\{\kappa,\tau\}}_{\BT_5}$
    und man sieht ebenfalls sofort, dass dann die entsprechende Einschränkung
    von~$\psi$ ein Inverses zu~$\phi$ liefert, d.\,h. $\phi$ ist ein
    Isomorphismus und dies zeigt die Behauptung.
    \\
\end{proof}

\begin{proof}[Beweis von \cref{ch1:RinBT6}]
    Die Lemmata \ref{ch1:HNNeBT4}, \ref{ch1:HNNeBT5} und
    \ref{ch1:HNNeBT6} zeigen den ersten Teil. Der zweite
    Teil folgt mittels $R\leq \BT \fprod R = \hBT_4$ und
    mehrmaliger Anwendung von \cref{ch0:hnnembedding} aus
    dem ersten Teil.\\
\end{proof}

\begin{thProposition}\label{ch1:BT6fp}
    Die Gruppe $\BT_6$ ist endlich präsentiert.
\end{thProposition}
%
\begin{proof}
    Zunächst sehen wir direkt an der Konstruktion von~$\BT_6$
    (Definitionen \ref{ch1:BT} und~\ref{ch1:BT6fromBT}), dass
    $\BT_6$ endlich erzeugt ist von
    \[ \{q,h\} \cup S\cup Q
        \cup \{r_i\Mid i\in I\}\cup\{x,\kappa,\tau\}
        \cup U \cup B \cup \{d,\sigma\}
    \]
    und dass
    \[ \underbrace{\Delta \cup \bigcup\nolimits_{j=1}^6 \Delta_j}_{
        \eqdef\widetilde\Delta\;\,\mathrlap{\text{endl. viele Rel.}}
        } \surround{\quad}\cup E
    \]
    alle Relationen von~$\BT_6$ umfasst. Um nun einzusehen,
    dass $\BT_6$ sogar endlich präsentiert ist, genügt es zu zeigen,
    dass die Relationen in~$E$ schon von denen in~$\widetilde\Delta$
    impliziert werden.
    
    Sei also $\omega_u\in E$ und seien $\omega\defeq\tr^U_A(\omega_u)$,
    $\omega_b\defeq\tr^U_B(\omega)$. Wir zeigen zuerst, dass
    $\kappa$ mit $\omega\inv d\omega$ kommutiert:
    \begin{align*}
        \kappa\omega\inv d\omega
        &=
        \kappa\omega\inv (\tau\inv\omega\omega\inv\tau) d\omega
        \\
        &\overset{\scriptscriptstyle(1)}=
        \omega\inv\tau\inv\omega\kappa \omega\inv
            (\sigma\inv\tau\sigma)\omega
        \\
        &\overset{\scriptscriptstyle(2)}=
        \omega\inv\tau\inv\sigma\inv
            \omega\kappa \omega\inv\tau\omega\sigma
        \\
        &\overset{\scriptscriptstyle(3)}=
        \omega\inv\tau\inv\sigma\inv
            \omega\omega\inv\tau\omega\kappa\sigma
        \\
        &\overset{\scriptscriptstyle(4)}=
        \omega\inv\tau\inv\sigma\inv\tau\sigma\omega\kappa
        \\
        &\overset{\scriptscriptstyle(5)}=
        \omega\inv d\omega\kappa
    , \end{align*}
    wobei wir Folgendes benutzt haben:
    \begin{enumerate}[(1), itemsep=0pt, topsep=0.5\baselineskip]
        \item
            \mycref{ch1:BTproperties:boone} und $\tau d =
            \sigma\inv\tau\sigma$ (aus $\Delta_6$)
        \item
            Aus den Relationen in~$\Delta_6$ folgt: $\sigma$ kommutiert
            mit~$\omega$ und mit~$\kappa$.
        \item
            \mycref{ch1:BTproperties:boone}
        \item
            Nochmal $\cu[\sigma, \omega] = 1$
            und $\cu[\sigma, \kappa] = 1$.
        \item
            Nochmal $\tau d = \sigma\inv\tau\sigma$ in Form von
            $\cu[\tau, \sigma] = d$.
    \end{enumerate}
    Nun erhalten wir:
    \begin{align*}
        \omega_b\omega_u
        &\overset{\scriptscriptstyle(1)}=
        \kappa\inv\omega_b\kappa
        \\
        &\overset{\scriptscriptstyle(2)}=
        \kappa\inv\omega_b d\kappa d'
        \\
        &\overset{\scriptscriptstyle(3)}=
        \kappa\inv \omega\inv d\omega \kappa d'
        \\
        &\overset{\scriptscriptstyle(4)}=
        \omega\inv d\omega d'
        \\
        &\overset{\scriptscriptstyle(5)}=
        \omega_b
    , \end{align*}
    wobei wir diesmal verwendet haben:
    \begin{enumerate}[leftmargin=2.155cm, itemsep=0pt, topsep=0.5\baselineskip]
        \item[(1),(3),(5)]
            \cref{ch1:conjugatesinBT5}
        \item[(2)]
            $\cu[d, \kappa] = 1$ (aus $\Delta_5$)
        \item[(4)]
            $\cu[\kappa, \omega\inv d\omega] = 1$
    \end{enumerate}
    Es folgt
    \[ \omega_u = 1  , \]
    und wir haben nur Relationen aus~$\widetilde\Delta$ benutzt, um dies
    zu zeigen.
    \\
\end{proof}

\begin{proof}[Beweis von \cref{ch1:higman}]
    Folgt direkt aus dem zweiten Teil von \cref{ch1:RinBT6}
    und \cref{ch1:BT6fp}.
    \\
\end{proof}
