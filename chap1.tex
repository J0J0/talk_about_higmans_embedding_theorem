%\section{Der Einbettungssatz von Higman}
\setcounter{section}{1}
\begin{thDef}[rekursiv präsentiert]
    Sei $R$ eine Gruppe. Dann ist $R$ \emph{rekursiv präsentiert},
    falls $R$ endlich erzeugt ist und es eine Präsentation
    $R = \gpres{u_1,\dots,u_m}{E}$ von~$R$ gibt, so dass
    jedes $\omega\in E$ ein positives Wort auf~$\{u_1,\dots,u_m\}$
    und die Menge~$E$ rekursiv aufzählbar ist.
\end{thDef}

\begin{thSatz}[Einbettungssatz von Higman]
    \label{ch1:higman}
    %
    Jede rekursiv präsentierte Gruppe kann in eine endlich präsentierte Gruppe
    eingebettet werden.
\end{thSatz}

\begin{thSetup}
    Sei ab jetzt eine rekursiv präsentierte Gruppe
    $R = \gpres{u_1,\dots,u_m}{E}$
    fixiert.
    \begin{itemize}
        \item
            Sei $U \defeq \{u_1,\dots,u_m\}$ und seien
            $A \defeq \{a_1,\dots,a_m\}$ sowie $B\defeq\{b_1,\dots,b_m\}$
            Mengen mit neuen Symbolen.
        \item
            Sei $\tr^U_A\colon U^*\to A^*$ der Monoid-Isomorphismus
            mit $u_j\mapsto a_j$ für alle $j\in\setOneto m$; seien
            $\tr^A_U,\tr^U_B,\tr^A_B,\tr^B_U$ analog definiert.
            Sei weiter $\Ea \defeq \tr^U_A(E)$.
        \item
            Sei $T$~eine Turingmaschine, die $\Ea$ aufzählt,
            sei $S = \{s_0,\dots,s_M\} \supset A$ ihr Eingabealphabet
            und sei $Q = \{q_0,\dots,q_N\}$ ihre Zustandsmenge (o.\,E.
            mit einzigem Endzustand~$q_0$).
    \end{itemize}
\end{thSetup}

\begin{thDefProposition}[Variante von Boones Gruppe]
    \label{ch1:BT}
    %
    Die Gruppe~$\BT$ ist gegeben durch
    \begin{itemize}
        \item Erzeuger:
                $\{q,h\} \cup S\cup Q
                \cup \{r_i\Mid i\in I\}\cup\{x,\kappa,\tau\}$
                {\small%
                (mit neuen Symbolen $r_i$ und $q,h,x,\kappa,\tau$)}.
        \smallskip
        \item Relationen:
                $\Delta_1 \cup \Delta_2 
                    \cup \Delta_3 \cup \Delta$,
                wobei
                \begin{align*}
                    \Delta_1 &\defeq \bigl\{ s\inv xs = x^2
                        \Mid s\in S_+ \bigr\}
                    \\[4pt]
                    \Delta_2 &\defeq \bigl\{ r_i\inv sxr_i = sx\inv
                        \Mid s\in S_+,\, i\in I \bigr\}
                        \cup \bigl\{
                        r_i\inv F_i^\# q_{i,1}G_ir_i = H_i^\# q_{i,2}K_i
                        \Mid i\in I \bigr\}
                    \\[4pt]
                    \Delta_3 &\defeq \bigl\{
                        \cu[\tau, q_1\inv hr_ih\inv q_1] \Mid i\in I
                        \bigr\} \cup \bigl\{
                        \cu[\tau, q_1\inv hxh\inv q_1] \bigr\}
                    \\[4pt]
                    \Delta &\defeq \bigl\{
                        \cu[\kappa, hr_ih\inv] \Mid i\in I
                        \bigr\} \cup \bigl\{
                        \cu[\kappa, hxh\inv] \bigr\}
                        \cup \bigl\{
                        \cu[\kappa,
                            hq\inv h\inv q_1\tau q_1\inv hqh\inv]
                        \bigr\}
                ; \end{align*}
    \end{itemize}
    dabei sind $F_i,G_i,H_i,K_i$ (von $T$ abhängige) positive Wörter
    auf~$S_+\defeq S\cup\{h\}$ und für
    $F = y_1^{\epsilon_1}\cdots y_n^{\epsilon_n}$ ist
    $F^\# = y_1^{-\epsilon_1}\cdots y_n^{-\epsilon_n}$.
    Für ein positives Wort~$w$ auf~$S$ gilt:
    \[ w\in\Ea \iff \cu[\kappa, w\inv\tau w] = 1 \in\BT
    . \]
\end{thDefProposition}

\smallskip
\begin{proof}[Beweisskizze von Proposition~\ref{ch1:BT}]
    Die Konstruktion von~$\BT$ (aus der zu $T$ assoziierten Halbgruppe)
    erfolgt analog zur Gruppe von Boone, die in den letzten Vorträgen
    verwendet wurde. Mit dem Übersetzungsschema
    \enquote{$\tau = q_1\inv hth\inv q_1$,
    $\kappa = hkh\inv$} sieht man auch leicht, dass~$\BT$ sogar isomorph
    zu Boones Gruppe ist.     
    Das \emph{Lemma von Boone} in Kombination mit der grundlegenden
    Eigenschaft der zu~$T$ assoziierten Halbgruppe liefert dann die
    gewünschte Aussage über positive Wörter auf~$S$.
    \\
\end{proof}

\begin{thDef}[Erweiterungen von \texorpdfstring{$\BT$}{B}]
    \label{ch1:BT6fromBT}
    %
    Wir definieren sukzessive Erweiterungen von~$\BT$ wie folgt:
    \begin{align*}
        \hBT_4 &\defeq \gadj{\BT}{U}{E} = \BT \fprod R
        \\
        \BT_4 &\defeq \gadj{\hBT_4}{B}{\Delta_4}
        \\
        \BT_5 &\defeq \gadj{\BT_4}{\{d\}}{\Delta_5}
        \\
        \BT_6 &\defeq \gadj{\BT_5}{\{\sigma\}}{\Delta_6}
    , \end{align*}
    wobei
    \begin{align*}
        \Delta_4 &\defeq \bigl\{
            \cu[b, u] \Mid b\in B,\, u\in U
            \bigr\} \cup \bigl\{
            \cu[b, a] \Mid b\in B,\, a\in A
            \bigr\}
            \cup \bigl\{
            b_j\inv\kappa b_j = \kappa u_j\inv \Mid j\in\J \bigr\}
        \\[2pt]
        \Delta_5 &\defeq \bigl\{ \cu[d, \kappa] \bigr\}
            \cup \bigl\{ d\inv a_jb_j d = a_j \Mid j\in\J \bigr\}
        \\[2pt]
        \Delta_6 &\defeq \bigl\{ \cu[\sigma, \kappa],\;
            \sigma\inv\tau\sigma = \tau d \bigr\}
            \cup \bigl\{ [\sigma, a] \Mid a\in A \bigr\}
    . \end{align*}
\end{thDef}

\begin{thProposition}[\texorpdfstring{$R\leq \BT_6$}%
                        {R als Untergruppe von B6}]\hfill
    \label{ch1:RinBT6}
    %
    \begin{enumerate}[1.]
        \item 
            Es entsteht $\BT_6$ aus $\hBT_4$ als eine Kette von
            HNN-Erweiterungen wie folgt:
            {\thickmuskip=10mu%
            \[ \hBT_4 \leq \BT_4 \leq \BT_5 \leq \BT_6  . \]}
        \item
            Es ist $R$ (vermöge der kanonischen Abbildung
            $R\to\BT_6$) eine Untergruppe von~$\BT_6$.
    \end{enumerate}
\end{thProposition}

\begin{thAufgabe}
    Zeige: Es ist $A\cup\{\kappa\}$ ein freies Erzeugendensystem der
    von dieser Menge in~$\BT$ erzeugten Gruppe.\\
    Hinweise: Benutze, dass $A$ ein freies Erzeugendensystem
    von~$\Spann{A}_{\BT}$ ist. Verwende dann eine geeignete
    HNN-Erweiterung und das \emph{Lemma von Britton}.
\end{thAufgabe}

\begin{thLemma}\label{ch1:conjugatesinBT5}
    Sei $\omega\in\Ea$ und seien $\omega_b \defeq \tr^A_B(\omega)$,
    $\omega_u \defeq \tr^A_U(\omega)$. Dann gilt in~$\BT_5$:
    \begin{align*}
        \kappa\inv \omega_b \kappa &= \omega_b\omega_u  \\
        d \omega d\inv &= \omega\omega_b
    . \end{align*}
\end{thLemma}

\begin{thAufgabe}
    Beweise \cref{ch1:conjugatesinBT5}.
\end{thAufgabe}

\begin{thProposition}[Endliche Präsentation von \texorpdfstring{$\BT_6$}{B6}]
    \label{ch1:BT6fp}
    %
    Die Gruppe $\BT_6$ ist endlich präsentiert.
\end{thProposition}

\begin{proof}[Beweis von \cref{ch1:higman}]
    %
    Der Einbettungssatz von Higman folgt nun aus dem zweiten Teil von
    \cref{ch1:RinBT6} und \cref{ch1:BT6fp}.
    \\
\end{proof}

\begin{thAufgabe}
    Zeige, dass in folgendem Sinne auch die Umkehrung von \cref{ch1:higman}
    gilt: Sei $G$ eine endlich erzeugte Gruppe. Wenn $G$ in eine endlich
    präsentierte Gruppe eingebettet werden kann, so ist $G$ schon rekursiv
    präsentiert.
\end{thAufgabe}
