\chapter{Der Einbettungssatz von Higman}
\begin{thDef}[rekursiv präsentiert]
    Sei $R$ eine Gruppe. Dann ist $R$ \emph{rekursiv präsentiert},
    falls $R$ endlich erzeugt ist und es eine Präsentation
    $R = \gpres{u_1,\dots,u_m}{E}$ von~$R$ gibt, so dass
    jedes $\omega\in E$ ein positives Wort auf~$\{u_1,\dots,u_m\}$
    und die Menge~$E$ rekursiv aufzählbar ist.
\end{thDef}

\begin{thSatz}[Einbettungssatz von Higman]
    \label{ch1:higman}
    %
    Jede rekursiv präsentierte Gruppe kann in eine endlich präsentierte Gruppe
    eingebettet werden.
\end{thSatz}

Den Rest des Skripts werden wir nun damit verbringen, diesen Satz zu beweisen.

\begin{thSetup}\hfill
    \begin{itemize}
    \item 
        Sei ab jetzt eine rekursiv präsentierte Gruppe
        \[ R = \gpres{u_1,\dots,u_m}{E} \]
        fixiert und sei $U \defeq \{u_1,\dots,u_m\}$.
    \item
        Aus notationellen Gründen führen wir eine weitere Menge
        $A \defeq \{a_1,\dots,a_m\}$ von neuen Symbolen ein und bezeichnen mit
        $\tra$ den eindeutigen Monoid-Isomorphismus (vom Monoid aller
        positiven Wörter auf $\{u_1,\dots,u_m\}$ in das Monoid aller positiven
        Wörter auf $\{a_1,\dots,a_m\}$) mit $u_j\mapsto a_j$ für alle
        $j\in\setOneto{m}$. Sei $\Ea \defeq \tra(E)$.
    \item
        Sei weiter $T$~eine Turingmaschine, die $\Ea$ aufzählt,\footnote{%
            So eine Turingmaschine~$T$ existiert offenbar, denn nach
            Voraussetzung gibt es eine Turingmaschine, die $E$ aufzählt,
            und $E\rightsquigarrow\Ea$ ist nur eine Umbenennung der
            Symbole.%
        }
        sei $S = \{s_0,\dots,s_M\} \supset A$ ihr Eingabealphabet
        (wobei $s_0$ das \enquote{leeres Band}-Zeichen sei) und sei
        $Q = \{q_0,\dots,q_N\}$ ihre Zustandsmenge. Wir nehmen ohne
        Einschränkung an, dass $q_0$ der einzige Endzustand sowie $q_1$ der
        Anfangszustand von~$T$ ist.
    \end{itemize}
\end{thSetup}

\begin{thProposition}[Halbgruppe assoziiert zu~$T$ {[Post]}]
    \label{ch1:Gamma}
    %
    Es gibt eine Halbgruppe~$\Gamma$ mit der Präsentation
    \begin{itemize}
        \item Erzeuger:
                $\{q,h\} \cup S\cup Q$ (mit neuen Symbolen $q,h\notin S\cup Q$),
        \item Relationen:
                $\{ F_iq_{i,1}G_i = H_iq_{i,2}K_i \Mid i\in I \}$,
                wobei $I$ eine endliche Indexmenge ist und für alle
                $i\in I$ gilt:
                $q_{i,1},q_{i,2}\in \{q\}\cup Q$ und $F_i,G_i,H_i,K_i$ sind
                (möglicherweise leere) positive Wörter auf $S\cup\{h\}$,
    \end{itemize}
    und folgender Eigenschaft: Ist $w$ ein positives Wort auf~$S$, so
    gilt $w\in E \iff hq_1wh = q \in\Gamma$.
\end{thProposition}
%
Siehe Rotman\cite{bookc:rotman95}. % TODO

\pagebreak[2]
Im Folgenden seien $\Gamma$ und alle zur Präsentation gehörenden Daten
wie in~\cref{ch1:Gamma} fest gewählt. Zur Vereinfachung der Notation setzen
wir außerdem $S_+ \defeq S\cup\{h\}$.

\begin{thDef}[Variante von Boones Gruppe]
    \label{ch1:BT}
    %
    Sei die Gruppe $\BT$ gegeben durch:
    \begin{itemize}
        \item Erzeuger:
                $\{q,h\} \cup S\cup Q
                \cup \{r_i\Mid i\in I\}\cup\{x,\kappa,\tau\}$
                (mit neuen Symbolen $r_i$ und $x,\kappa,\tau$).
        \item Relationen:
                $\Delta_1 \cup \Delta_2 
                    \cup \Delta_3 \cup \Delta$,
                wobei
                \begin{align*}
                    \Delta_1 &\defeq \bigl\{ s\inv xs = x^2
                        \Mid s\in S_+ \bigr\}
                    \\[4pt]
                    \Delta_2 &\defeq \bigl\{ r_i\inv sxr_i = sx\inv
                        \Mid s\in S_+,\, i\in I \bigr\}
                    \\
                    &\qquad\cup \bigl\{
                        r_i\inv F_i^\# q_{i,1}G_ir_i = H_i^\# q_{i,2}K_i
                        \Mid i\in I \bigr\}
                    \\[4pt]
                    \Delta_3 &\defeq \bigl\{
                        \cu[\tau, q_1\inv hr_ih\inv q_1] \Mid i\in I
                        \bigr\} \cup \bigl\{
                        \cu[\tau, q_1\inv hxh\inv q_1] \bigr\}
                    \\[4pt]
                    \Delta &\defeq \bigl\{
                        \cu[\kappa, hr_ih\inv] \Mid i\in I
                        \bigr\} \cup \bigl\{
                        \cu[\kappa, hxh\inv] \bigr\}
                    \\
                    &\qquad\cup \bigl\{
                        \cu[\kappa,
                            hq\inv h\inv q_1\tau q_1\inv hqh\inv]
                    \bigr\}
                . \end{align*}
    \end{itemize}
    Dabei ist für ein Wort~$F = y_1^{\epsilon_1}\cdots y_n^{\epsilon_n}$
    auf~$S_+$ (mit $\epsilon_j\in\{\pm1\}$ für alle $j\in\setOneto n$)
    das Wort $F^\#$ definiert als $y_1^{-\epsilon_1}\cdots y_n^{-\epsilon_n}$.
\end{thDef}

\begin{thProposition}[Eigenschaften von \texorpdfstring{$\BT$}{B}]\hfill
    \label{ch1:BTproperties}
    %
    \begin{enumerate}[1.]
        \item\label{ch1:BTproperties:boone}
            Die Gruppe~$\BT$ erfüllt folgende Eigenschaft:
            Ist $w$ ein positives Wort auf~$S$, so gilt
            \[ w\in E \iff \cu[\kappa, w\inv\tau w] = 1 \in\BT
            . \]

        \item\label{ch1:BTproperties:hnn}
            Es entsteht $\BT$ als eine Kette von HNN-Erweiterungen
            {\thickmuskip=10mu%
            \[ \BT_0 \leq \BT_1 \leq \hBT_2 \leq \BT_2 \leq \BT_3 \leq \BT
            , \]}
            wobei
            \begin{align*}
                \BT_0 &\defeq \gpres{x}{\emptyset}
                \\
                \BT_1 &\defeq \gadj{\BT_0}{S_+}{\Delta_1}
                \\
                \hBT_2 &\defeq \gadj{\BT_1}{Q\cup\{q\}}{\emptyset}
                         = \BT_1 \fprod \gpres{Q\cup\{q\}}{\emptyset}
                \\
                \BT_2 &\defeq \gadj{\hBT_2}{\{r_i\Mid i\in I\}}{\Delta_2}
                \\
                \BT_3 &\defeq \gadj{\BT_2}{\{\tau\}}{\Delta_3}
                \\
                \BT &\mathrel{\makebox[\widthof{$\defeq$}]{\hfill$=$}}
                        \gadj{\BT_3}{\{\kappa\}}{\Delta}
            \end{align*}
            und die stabilen Buchstaben jeweils die
            neu hinzugefügten Symbole %TODO: \ref zur Def. von \gadj..!?
            sind.
    \end{enumerate}
\end{thProposition}

\begin{thBemerkung}[Notation bei Rotman]
    Im Buch von Rotman\cite{bookc:rotman95} lauten die Bezeichnungen
    der Gruppen $\BT_3$ bzw. $\BT$ leicht anders:
    $\mathscr{B}_3'(T)$ bzw. $\mathscr{B}'(T)$. Die Gruppen
    $\hBT_2$ und $\hBT_4$ (s.\,u.) haben bei Rotman keine explizite
    Bezeichnung. Ansonsten wurde die Notation so gewählt, dass die
    Nummerierung der Gruppen mit der von Rotman übereinstimmt.
\end{thBemerkung}

\begin{proof}[Beweisskizze von \cref{ch1:BTproperties}]
    Zu~\ref{ch1:BTproperties:boone}: Wir betrachten die bei Rotman 
    in Kapitel~12 überhalb von Lemma~12.7 definierte Gruppe~$\mathscr{B}(T)$
    (wobei zu beachten ist, dass das dortige $s_{M-1}$ unserem $s_M$
    und das dortige $s_M$ unserem $h$ entspricht).
    Durch Vergleich der Präsentationen der Gruppen $\mathscr{B}(T)$
    und~$\BT$ sieht man leicht mithilfe der universellen Eigenschaft
    von durch Erzeuger und Relationen gegeben Gruppen ein, dass man
    zueinander inverse Gruppenhomomorphismen $\phi\colon\mathscr{B}(T)
    \to\BT$ und $\psi\colon\BT\to\mathscr{B}(T)$ mit
    \begin{align*}
        \phi(\tau)   &= q_1\inv hth\inv q_1
        ,&
        \psi(t)      &= h\inv q_1\tau q_1\inv h
        \\
        \phi(\kappa) &= hkh\inv
        ,&
        \psi(k)      &= h\inv\kappa h
    \end{align*}
    konstruieren kann. Das \emph{Lemma von Boone}
    (Rotman\cite[Lemma~12.7]{bookc:rotman95}) liefert dann
    (in der Notation von Rotman für $\Sigma \equiv h\inv q_1wh$):
    Für ein positives Wort~$w$ auf~$S$ gilt:
    \[ hq_1wh = q \in\Gamma
        \qiffq
        \cu[k, (h\inv q_1wh)\inv t (h\inv q_1wh)] = 1
        \in\mathscr{B}(T)
    . \]
    Unter dem Isomorphismus~$\phi$ wird die rechte Gleichung
    dann gerade zu $\cu[\kappa, w\inv\tau w] = 1 \in\BT$
    und mit \cref{ch1:Gamma} folgt die erste Behauptung.
    
    Zu~\ref{ch1:BTproperties:hnn}: Die letzten beiden Erweiterungen
    sind HNN-Erweiterungen nach \cref{ch0:simpleHNN}. Dass $\hBT_2$
    eine HNN-Erweiterung mit Basis~$\BT_2$ ist, ist klar (wähle
    $M_{\tilde q} = \emptyset$ für alle $\tilde q\in Q\cup\{q\}$).
    Weiter induziert $x\mapsto x^2$ klarerweise einen Isomorphismus
    zwischen den unendlich zyklischen Gruppen $\scriptstyle\Spann{x}_{\BT_0}$
    und $\scriptstyle\Spann{x^2}_{\BT_0}$, was zeigt, dass $\BT_1$ eine
    HNN-Erweiterung mit Basis~$\BT_0$ ist. Es bleibt zu zeigen,
    dass $\BT_2$ eine HNN-Erweiterung mit Basis~$\hBT_2$ ist.
    Dazu zeigt man (für jedes $i\in I$), dass $\{F_i^\#q_{i,1}G_i\}\cup
    S_+x$ ein freies Erzeugendensystem der von dieser Menge
    in~$\hBT_2$ erzeugten Untergruppe ist und analog für
    $\{H_i^\#q_{i,2}K_i\}\cup S_+x\inv$, womit man leicht die
    gewünschten Isomorphismen konstruiert. 
    \\
\end{proof}

\begin{thDef}[Erweiterungen von \texorpdfstring{$\BT$}{B}]
    Sei $B \defeq \{b_1,\dots,b_m\}$ eine Menge neuer Symbole
    und seien weiter $d,\sigma$ neue Symbole. Dann definieren
    wir sukzessive Erweiterungen von~$\BT$ wie folgt:
    \begin{align*}
        \hBT_4 &\defeq \gadj{\BT}{U}{E} = \BT \fprod R
        \\
        \BT_4 &\defeq \gadj{\hBT_4}{B}{\Delta_4}
        \\
        \BT_5 &\defeq \gadj{\BT_4}{\{d\}}{\Delta_5}
        \\
        \BT_6 &\defeq \gadj{\BT_5}{\{\sigma\}}{\Delta_6}
    , \end{align*}
    wobei
    \begin{align*}
        \Delta_4 &\defeq \bigl\{
            \cu[b, u] \Mid b\in B,\, u\in U
            \bigr\} \cup \bigl\{
            \cu[b, a] \Mid b\in B,\, a\in A
            \bigr\}
        \\
        &\qquad\cup \bigl\{
            b_j\inv\kappa b_j = \kappa u_j\inv \Mid j\in\J \bigr\}
        \\[2pt]
        \Delta_5 &\defeq \bigl\{ \cu[\kappa, d] \bigr\}
            \cup \bigl\{ d\inv a_jb_j d = a_j \Mid j\in\J \bigr\}
        \\[2pt]
        \Delta_6 &\defeq \bigl\{ \cu[\sigma, \kappa],\;
            \sigma\inv\tau\sigma = \tau d \bigr\}
            \cup \bigl\{ [\sigma, a] \Mid a\in A \bigr\}
    . \end{align*}
\end{thDef}

\pagebreak[2]
\begin{thProposition}[\texorpdfstring{$R\leq \BT_6$}%
                        {R als Untergruppe von B6}]\hfill
    \label{ch1:RinB6}
    %
    \begin{enumerate}[1.]
        \item 
            Es entsteht $\BT_6$ aus $\hBT_4$ als eine Kette von
            HNN-Erweiterungen wie folgt:
            {\thickmuskip=10mu%
            \[ \hBT_4 \leq \BT_4 \leq \BT_5 \leq \BT_6  . \]}
        \item
            Es ist $R$ (vermöge der kanonischen Abbildung
            $R\to\BT_6$) eine Untergruppe von~$\BT_6$.
    \end{enumerate}
\end{thProposition}

\begin{thLemma}
    Es sind $A\cup\{\kappa\}$ bzw. $A\cup\{\tau\}$ freie
    Erzeugendensysteme der von diesen Mengen in $\hBT_4$
    erzeugten Untergruppen.
\end{thLemma}

\begin{thLemma}\label{ch1:HNNeBT4}
    Es ist $\BT_4$ eine HNN-Erweiterung mit Basis~$\hBT_4$.
\end{thLemma}

\begin{thLemma}\label{ch1:HNNeBT5}
    Es ist $\BT_5$ eine HNN-Erweiterung mit Basis~$\BT_4$.
\end{thLemma}

\begin{thLemma}
    Die von $A\cup\{\kappa,\tau\}$ in $\BT$ erzeugte Untergruppe
    besitzt folgende Präsentation:
    \[ \gpres[{[\big]}]{A\cup\{\kappa,\tau\}}%
                   {\{ \cu[\kappa, \omega\inv\tau\omega]
                        \nMid \omega\in\Ea \}}
    . \]
\end{thLemma}

TODO: Beweisskizze (Schritt~1 ausführlich, Schritte des
Widerspruchsbeweises danach nur sehr knapp, eventuell einen
davon ausgeführt)

\begin{thLemma}\label{ch1:HNNeBT6}
    Es ist $\BT_6$ eine HNN-Erweiterung mit Basis~$\BT_5$.
\end{thLemma}

\begin{proof}[Beweis von \cref{ch1:RinB6}]
    Die Lemmata \ref{ch1:HNNeBT4}, \ref{ch1:HNNeBT5} und
    \ref{ch1:HNNeBT6} zeigen den ersten Teil. Der zweite
    Teil folgt mittels $R\leq \BT \fprod R = \hBT_4$ und
    mehrmaliger Anwendung von \cref{ch0:hnnembedding} aus
    dem ersten Teil.\\
\end{proof}

\begin{thProposition}\label{ch1:B6fp}
    Die Gruppe $\BT_6$ ist endlich präsentiert.
\end{thProposition}

\begin{proof}[Beweis von \cref{ch1:higman}]
    Folgt direkt aus dem zweiten Teil von \cref{ch1:RinB6}
    und \cref{ch1:B6fp}.
    \\
\end{proof}
