
\chapter{Vorwort, Notation und Vorbereitungen}
\section{Einleitung}
(\dots) % TODO

\section{Notation}
In diesem Skript wird folgende Notation verwendet:
\begin{itemize}
    \item
        Echt enthalten wird durch $\subsetneq$ gekennzeichnet,
        enthalten oder gleich durch $\subset$.
    
    %\item
    %    Die \emph{natürlichen Zahlen $\N$} beginnen mit $0$.
        
    \item
        Mit $\gadj{G}{X}{Y}$ bezeichnen wir das Hinzufügen von Erzeugern und
        Relationen zu einer Gruppe, d.\,h.: Ist $\gpres{X'}{Y'}$ eine
        Präsentation von~$G$, so besitzt $\gadj{G}{X}{Y}$ die Präsentation
        $\gpres{X'\sqcup X}{Y'\cup Y}$. Auch wenn dies generell unabhängig
        von der Wahl der Präsentation von~$G$ ist, so benutzen wir im Folgenden
        stets die aus dem Kontext ersichtliche Präsentation von~$G$.
\end{itemize}

\section{Erinnerung}
\begin{thProposition}
    \label{ch0:hnnembedding}
    %
    Sei $G$ eine HNN-Erweiterung mit Basis~$G'$. Dann ist der kanonische
    Homomorphismus $G'\to G$ injektiv, d.\,h. wir können $G'$ als
    Untergruppe von~$G$ auffassen.
\end{thProposition}

\begin{thLemma}[Lemma von Britton]
    \label{ch0:brittonslemma}
    %
    Sei $G$ eine HNN-Erweiterung und sei $w$ ein Wort aus~$G$,
    das mindestens einen stabilen Buchstaben enthält und für
    das $w=1$ in~$G$ gilt. Dann enthält $w$ einen pinch als Teilwort.
\end{thLemma}
