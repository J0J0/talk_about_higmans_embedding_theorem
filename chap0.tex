
\chapter{Vorwort, Notation und Vorbereitungen}
\section{Einleitung}
Im letzten Vortrag haben wir gesehen, wie man das \emph{Lemma von Boone}
beweist. Die von Boone konstruierte Gruppe werden wir nun verwenden,
um den \emph{Einbettungssatz von Higman} \pcref{ch1:higman} zu beweisen.
Der Beweis in dieser Form geht auf Aanderaa zurück\,\cite{paper:aanderaa73},
aber wir halten uns im Folgenden an die Fassung, wie sie bei
Rotman\,\cite[Kapitel~12]{bookc:rotman95} zu finden ist.

\section{Notation}
In diesem Skript wird folgende Notation verwendet:
\begin{itemize}
    \item
        Echt enthalten wird durch $\subsetneq$ gekennzeichnet,
        enthalten oder gleich durch $\subset$.
        
    \item
        Mit $\gadj{G}{X}{Y}$ bezeichnen wir das Hinzufügen von Erzeugern und
        Relationen zu einer Gruppe, d.\,h.: Ist $\gpres{X'}{Y'}$ eine
        Präsentation von~$G$, so besitzt $\gadj{G}{X}{Y}$ die Präsentation
        $\gpres{X'\sqcup X}{Y'\cup Y}$. Wenn aus dem Kontext klar ist, welche
        Präsentation von~$G$ gemeint ist, sprechen wir auch einfach von
        \enquote{der Präsentation $\gadj{G}{X}{Y}$}.
        
    \item
        Wenn wir über Wörter~$w,w'$ (auf Erzeugern) von (Halb-)Gruppen sprechen,
        so steht $w \equiv w'$ für \enquote{die Wörter sind Symbol für Symbol
        identisch} und $w = w'$ für \enquote{die Wörter repräsentieren in der
        (Halb-)Gruppe dasselbe Element}.

    \item
        Für eine Menge von Symbolen~$S$ bezeichnet $S^*$ die Menge aller Wörter
        auf~$S$ (einschließlich der (formal) Inversen zu den Symbolen aus~$S$).
        Ein Wort, das nur Symbole aus~$S$ enthält, nennen wir
        \emph{positives Wort}.
\end{itemize}

\section{Erinnerung: HNN-Erweiterungen}
\begin{thDef}[HNN-Erweiterung]
    Seien $G,G'$ Gruppen und $P$ eine nicht-leere Menge. Dann ist \emph{$G$
    eine HNN-Erweiterung mit Basis~$G'$ und stabilen Buchstaben~$P$}, falls
    gilt:
    Es gibt eine Präsentation $\gpres{X}{Y}$ von~$G'$ (mit $X\cap P =
    \emptyset$) und für jedes $p\in P$ eine Menge
    $M_p \subset G'\times G'$, so dass folgende Eigschaften erfüllt sind:
    \begin{itemize}
        \item
            Es ist
            \[ \gadj[{[\big]}]{G'}{P}%
                {\{ p\inv ap = b \nMid p\in P,\, (a,b)\in M_p \}}
            \]
            eine Präsentation von~$G$.
        \item
            Für alle $p\in P$ gibt es einen Isomorphismus zwischen den
            von $\{ a \Mid (a,b)\in M_p\}$ und $\{ b \Mid (a,b)\in M_p\}$
            in~$G'$ erzeugten Untergruppen mit $a\mapsto b$ für alle
            $(a,b)\in M_p$.
    \end{itemize}
\end{thDef}

\begin{thBemerkung}[Kommutator-Relationen vs. HNN-Erweiterungen]
    \label{ch0:simpleHNN}
    %
    Haben wir
    \[ G \cong \gadj[{[\big]}]{G'}{\{p\}}%
                {\{ \cu[p, w] \Mid w\in W \}}
    \]
    für eine Menge von Wörtern~$W$ aus~$G'$, so ist die Kommutator-Relation
    $\cu[p, w] = 1$ offenbar äquivalent zu $p\inv w p = w$ und die
    Identität liefert einen Automorphismus der von $W$ in~$G'$ erzeugten
    Untergruppe. Insbesondere ist $G$ also eine HNN-Erweiterung mit Basis~$G'$
    und stabilem Buchstaben~$p$. (Eine analoge Aussage gilt auch für
    HNN-Erweiterungen mit mehreren Buchstaben.)
\end{thBemerkung}

\begin{thProposition}
    \label{ch0:hnnembedding}
    %
    Sei $G$ eine HNN-Erweiterung mit Basis~$G'$. Dann ist der kanonische
    Homomorphismus $G'\to G$ injektiv, d.\,h. wir können $G'$ als
    Untergruppe von~$G$ auffassen.
\end{thProposition}

\begin{thProposition}
    \label{ch0:hnnstablesfree}
    %
    Sei $G$ eine HNN-Erweiterung mit stabilen Buchstaben~$P$. Dann ist
    die von~$P$ in~$G$ erzeugte Untergruppe frei mit Basis~$P$.
\end{thProposition}

\begin{thDef}[pinch]
    Sei $G$ eine HNN-Erweiterung mit Basis~$G'$ und stabilen Buchstaben~$P$.
    Sei $p\in P$. Ein \emph{$p$-pinch} ist ein Wort $p^\epsilon w p^{-\epsilon}$
    mit $\epsilon\in\{\pm1\}$ und $w\in G'$, so dass
    \begin{itemize}[topsep=0pt]
        \item im Fall $\epsilon = -1$ gilt:
            $p\inv wp\in\Spann{\{ a \Mid (a,b)\in M_p\}}_{G'}$.
        \item im Fall $\epsilon = +1$ gilt:
            $pwp\inv\in\Spann{\{ b \Mid (a,b)\in M_p\}}_{G'}$.
    \end{itemize}
\end{thDef}

\begin{thLemma}[Lemma von Britton]
    \label{ch0:brittonslemma}
    %
    Sei $G$ eine HNN-Erweiterung und sei $w$ ein Wort aus~$G$,
    das mindestens einen stabilen Buchstaben enthält und für
    das $w=1$ in~$G$ gilt. Dann enthält $w$ einen pinch als Teilwort.
\end{thLemma}

Beweise der Propositionen und des Lemmas:
siehe Rotman\,\cite[Kapitel~11]{bookc:rotman95}.
